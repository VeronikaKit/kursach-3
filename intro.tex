
\section*{Введение}
\addcontentsline{toc}{section}{Введение}
Единый государственный экзамен (ЕГЭ)~— централизованно проводимый в Российской
Федерации экзамен в средних учебных заведениях — школах, лицеях и гимназиях,
форма проведения ГИА (Государственной Итоговой Аттестации) по образовательным программам среднего общего образования.
Служит одновременно выпускным экзаменом из школы и вступительным экзаменом в вузы.

За два года подготовки к ЕГЭ школьники сталкиваются с дефицитом заданий для подготовки. 
А учителя со списыванием ответов при решении задач экзамена учениками. 
Также в в конце 2021 года в список заданий ЕГЭ были добавлены новые задания под номером 9, 
количество которых для прорешивания очень мало. Были добавлены и текстовые задания в оганиченном количестве.
Проект «Час ЕГЭ» позволяет решить все эти проблемы. «Час ЕГЭ» — компьютерный образовательный проект, 
разрабатываемый при математическом факультете ВГУ в рамках «OpenSource кластера» и предназначенный для помощи учащимся старших классов подготовиться к тестовой части единого государственного экзамена. 
Задания в «Час ЕГЭ» генерируются случайным образом по специализированным алгоритмам, называемых шаблонами, 
каждый из которых охватывает множество вариантов соответствующей ему задачи. 
Для пользователей предназначены четыре оболочки (режима работы): «Случайное задание», «Тесты на печать», «Полный тест» и «Мини-интеграция». 
«Час ЕГЭ» является полностью открытым (код находится под лицензией GNU GPL 3.0) и бесплатным. 
В настоящее время в проекте полностью реализованы тесты по математике с кратким ответом (бывшая «часть В»). 
Планируется с течением времени включить в проект тесты по другим предметам школьной программы.

Первая глава этой работы посвящена реализация алгоритмов на базе проекта Час ЕГЭ.

Вторая глава представляет решение проблемы нехватки задач ЕГЭ и ОГЭ; шаблоны задач, написанные автором и примеры генерации.