\section{Вклад автора в расширение каталога}\label{2sect}
В это главе решается проблема нехватки заданий для подготовки к ЕГЭ, а также приводятся аргументы в пользу программного написания шаблонов для подготовки к ЕГЭ.

\subsection{Преимущества программной генерации заданий}
На примере предыдущей задачи было явно показано превосходство шаблонов над заданиями из Открытого Банка Заданий, а именно:
\begin{enumerate}
	\item Большое количество разнообразных задач одного типа. 
	\item Простота и скорость написания шаблонов. 
    \item Невозможность нахождения учащимися ответов на задачи.
\end{enumerate}

\subsection{Вклад автора в расширение каталога}

\lstinputlisting[caption = 26586.js, label={lst:26586}, escapechar=|]{code/26586.js}
\subsubsection*{Примеры генерируемых задач 26586.js}

\par{Моторная лодка прошла против течения реки $160$ км и вернулась в пункт отправления, затратив на обратный путь на $1$ час меньше. Вычислите скорость моторной лодки в неподвижной воде, если скорость течения равна $4$ км/ч. Ответ дайте в км/ч.}{36}{26586.js}
\par{Байдарка прошла против течения реки $96$ км и вернулась в пункт отправления, затратив на обратный путь на $4$ часа меньше. Определите скорость байдарки в неподвижной воде, если скорость течения равна $1$ км/ч. Ответ дайте в км/ч.}{7}{26586.js}
\par{Яхта прошла против течения реки $180$ км и вернулась в город отправления, затратив на обратный путь на $1$ час меньше. Найдите скорость яхты в неподвижной воде, если скорость течения равна $1$ км/ч. Ответ дайте в км/ч.}{19}{26586.js}

\lstinputlisting[caption = 26588.js, label={lst:26588}, escapechar=|]{code/26588.js}
\subsubsection*{Примеры генерируемых задач 26588.js}   

\par{Байдарка проходит по течению реки до деревни назначения $18$ км и после стоянки возвращается в деревня отправления. Определите скорость течения, если скорость байдарки в неподвижной воде равна $24$ км/ч, стоянка длится $1$ час, а в деревня отправления байдарка возвращается через $3$ часа. Ответ дайте в км/ч.}{12}{26588.js}
\par{Баржа проходит по течению реки до деревни назначения $160$ км и после стоянки возвращается в деревня отправления. Определите скорость течения, если скорость баржи в неподвижной воде равна $26$ км/ч, стоянка длится $2$ часа, а в деревня отправления баржа возвращается через $15$ часов. Ответ дайте в км/ч.}{6}{326588.js}
\par{Корабль проходит по течению реки до города назначения 63 км и после стоянки возвращается в город отправления. Определите скорость течения, если скорость корабля в неподвижной воде равна 16 км/ч, стоянка длится 2 часа, а в город отправления корабль возвращается через 10 часов. Ответ дайте в км/ч.}{2}{26588.js}

\lstinputlisting[caption = 27482.js, label={lst:27482}, escapechar=|]{code/27482.js}
\subsubsection*{Примеры генерируемых задач 27482.js}   

\par{Пристани $R$ и $C$ расположены на озере, расстояние между ними равно $120$ км. Баржа отправилась с постоянной скоростью из $R$ в $C$. Через два дня после прибытия она отправилась тем же путём обратно со скоростью на $4$ км/ч больше прежней, сделав по пути остановку на $1$ час. В результате она затратила на обратный путь столько же времени, сколько на путь из $R$ в $C$. Найдите скорость баржи на пути из $R$ в $C$. Ответ дайте в км/ч.}{20}{27482.js}
\par{Пристани $J$ и $L$ расположены на водохранилище, расстояние между ними равно $168$ км. Моторная лодка отправилась с постоянной скоростью из $J$ в $L$. На следующий день после прибытия она отправилась тем же путём обратно со скоростью на $12$ км/ч больше прежней, сделав по пути остановку на $7$ часов. В результате она затратила на обратный путь столько же времени, сколько на путь из $J$ в $L$. Найдите скорость моторной лодки на пути из $J$ в $L$. Ответ дайте в км/ч.}{12}{27482.js}
\par{Пристани $M$ и $E$ расположены на водохранилище, расстояние между ними равно $136$ км. Моторная лодка отправилась с постоянной скоростью из $M$ в $E$. Через два дня после прибытия она отправилась тем же путём обратно со скоростью на $17$ км/ч больше прежней, сделав по пути остановку на $4$ часа. В результате она затратила на обратный путь столько же времени, сколько на путь из $M$ в $E$. Вычислите скорость моторной лодки на пути из $M$ в $E$. Ответ дайте в км/ч.}{17}{27482.js}

\lstinputlisting[caption = 111557.js, label={lst:111557}, escapechar=|]{code/111557.js}
\subsubsection*{Примеры генерируемых задач 111557.js}

\par{Яхта прошла против течения реки $140$ км и вернулась в пункт отправления, затратив на обратный путь на $8$ часов меньше. Найдите скорость яхты в неподвижной воде, если скорость течения равна $1$ км/ч. Ответ дайте в км/ч.}{6}{111557.js}
\par{Лодка прошла против течения реки $21$ км и вернулась в пункт отправления, затратив на обратный путь на $9$ часов меньше. Вычислите скорость лодки в неподвижной воде, если скорость течения равна $6$ км/ч. Ответ дайте в км/ч.}{8}{111557.js}
\par{Яхта прошла против течения реки $120$ км и вернулась в пункт отправления, затратив на обратный путь на $2$ часа меньше. Определите скорость яхты в неподвижной воде, если скорость течения равна $8$ км/ч. Ответ дайте в км/ч.}{32}{111557.js}

\lstinputlisting[caption = 323375.js, label={lst:323375}, escapechar=|]{code/323375.js}
\subsubsection*{Примеры генерируемых задач 323375.js}   

\par{Корабль в $10:00$ вышел по течению реки из пункта $A$ в пункт $B$, расположенный в $45$ км от $A$. Пробыв в пункте $B$ $3$ часа, корабль отправился назад и вернулся в пункт $A$ в $22:00$ того же дня. Определите собственную скорость корабля(в км/ч), если известно, что скорость течения реки $12$ км/ч. Ответ дайте в км/ч.}{18}{323375.js}
\par{Теплоход в $1:00$ вышел по течению реки из пункта $A$ в пункт $B$, расположенный в $95$ км от $A$. Пробыв в пункте $B$ $2$ часа, теплоход отправился назад и вернулся в пункт $A$ в $15:00$ того же дня. Найдите собственную скорость теплохода(в км/ч), если известно, что скорость течения реки $14$ км/ч. Ответ дайте в км/ч.}{24}{323375.js}
\par{ Теплоход в $6:00$ вышел по течению реки из пункта $A$ в пункт $B$, расположенный в $4$ км от $A$. Пробыв в пункте $B$ $2$ часа, теплоход отправился назад и вернулся в пункт $A$ в $9:00$ того же дня. Определите собственную скорость теплохода(в км/ч), если известно, что скорость течения реки $3$ км/ч. Ответ дайте в км/ч.}{9}{323375.js}	
