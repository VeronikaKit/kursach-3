\section{Вклад автора в расширение каталога}\label{2sect}
В это главе решается проблема нехватки заданий для подготовки к ЕГЭ, а также приводятся аргументы в пользу программного написания шаблонов для подготовки к ЕГЭ.

\subsection{Преимущества программной генерации заданий}
На примере предыдущей задачи было явно показано превосходство шаблонов над заданиями из Открытого Банка Заданий, а именно:
\begin{enumerate}
	\item Большое количество разнообразных задач одного типа. 
	\item Простота и скорость написания шаблонов. 
    \item Невозможность нахождения учащимися ответов на задачи.
\end{enumerate}

\subsection{Вклад автора в расширение каталога}

\lstinputlisting[caption = 26586.js, label={lst:26586}, escapechar=|]{code/26586.js}
\subsubsection*{Примеры генерируемых задач 26586.js}

\par{Моторная лодка прошла против течения реки $160$ км и вернулась в пункт отправления, затратив на обратный путь на $1$ час меньше. Вычислите скорость моторной лодки в неподвижной воде, если скорость течения равна $4$ км/ч. Ответ дайте в км/ч. }{ 36 }{26586.js}
\par{Байдарка прошла против течения реки $96$ км и вернулась в пункт отправления, затратив на обратный путь на $4$ часа меньше. Определите скорость байдарки в неподвижной воде, если скорость течения равна $1$ км/ч. Ответ дайте в км/ч. }{ 7 }{26586.js}
\par{Яхта прошла против течения реки $180$ км и вернулась в город отправления, затратив на обратный путь на $1$ час меньше. Найдите скорость яхты в неподвижной воде, если скорость течения равна $1$ км/ч. Ответ дайте в км/ч. }{ 19 }{26586.js}

\lstinputlisting[caption = 26588.js, label={lst:26588}, escapechar=|]{code/26588.js}
\subsubsection*{Примеры генерируемых задач 26588.js}   

\par{Байдарка проходит по течению реки до деревни назначения $18$ км и после стоянки возвращается в деревня отправления. Определите скорость течения, если скорость байдарки в неподвижной воде равна $24$ км/ч, стоянка длится $1$ час, а в деревня отправления байдарка возвращается через $3$ часа. Ответ дайте в км/ч. }{ 12 }{26588.js}
\par{Баржа проходит по течению реки до деревни назначения $160$ км и после стоянки возвращается в деревня отправления. Определите скорость течения, если скорость баржи в неподвижной воде равна $26$ км/ч, стоянка длится $2$ часа, а в деревня отправления баржа возвращается через $15$ часов. Ответ дайте в км/ч. }{ 6 }{26588.js}
\par{Корабль проходит по течению реки до города назначения 63 км и после стоянки возвращается в город отправления. Определите скорость течения, если скорость корабля в неподвижной воде равна 16 км/ч, стоянка длится 2 часа, а в город отправления корабль возвращается через 10 часов. Ответ дайте в км/ч. }{ 2 }{26588.js}

\lstinputlisting[caption = 27482.js, label={lst:27482}, escapechar=|]{code/27482.js}
\subsubsection*{Примеры генерируемых задач 27482.js}   

\par{Пристани $R$ и $C$ расположены на озере, расстояние между ними равно $120$ км. Баржа отправилась с постоянной скоростью из $R$ в $C$. Через два дня после прибытия она отправилась тем же путём обратно со скоростью на $4$ км/ч больше прежней, сделав по пути остановку на $1$ час. В результате она затратила на обратный путь столько же времени, сколько на путь из $R$ в $C$. Найдите скорость баржи на пути из $R$ в $C$. Ответ дайте в км/ч. }{ 20 }{27482.js}
\par{Пристани $J$ и $L$ расположены на водохранилище, расстояние между ними равно $168$ км. Моторная лодка отправилась с постоянной скоростью из $J$ в $L$. На следующий день после прибытия она отправилась тем же путём обратно со скоростью на $12$ км/ч больше прежней, сделав по пути остановку на $7$ часов. В результате она затратила на обратный путь столько же времени, сколько на путь из $J$ в $L$. Найдите скорость моторной лодки на пути из $J$ в $L$. Ответ дайте в км/ч. }{ 12 }{27482.js}
\par{Пристани $M$ и $E$ расположены на водохранилище, расстояние между ними равно $136$ км. Моторная лодка отправилась с постоянной скоростью из $M$ в $E$. Через два дня после прибытия она отправилась тем же путём обратно со скоростью на $17$ км/ч больше прежней, сделав по пути остановку на $4$ часа. В результате она затратила на обратный путь столько же времени, сколько на путь из $M$ в $E$. Вычислите скорость моторной лодки на пути из $M$ в $E$. Ответ дайте в км/ч. }{ 17 }{27482.js}

\lstinputlisting[caption = 111557.js, label={lst:111557}, escapechar=|]{code/111557.js}
\subsubsection*{Примеры генерируемых задач 111557.js}

\par{Яхта прошла против течения реки $140$ км и вернулась в пункт отправления, затратив на обратный путь на $8$ часов меньше. Найдите скорость яхты в неподвижной воде, если скорость течения равна $1$ км/ч. Ответ дайте в км/ч. }{ 6 }{111557.js}
\par{Лодка прошла против течения реки $21$ км и вернулась в пункт отправления, затратив на обратный путь на $9$ часов меньше. Вычислите скорость лодки в неподвижной воде, если скорость течения равна $6$ км/ч. Ответ дайте в км/ч. }{ 8 }{111557.js}
\par{Яхта прошла против течения реки $120$ км и вернулась в пункт отправления, затратив на обратный путь на $2$ часа меньше. Определите скорость яхты в неподвижной воде, если скорость течения равна $8$ км/ч. Ответ дайте в км/ч. }{ 32 }{111557.js}

\lstinputlisting[caption = 323375.js, label={lst:323375}, escapechar=|]{code/323375.js}
\subsubsection*{Примеры генерируемых задач 323375.js}   

\par{Корабль в $10:00$ вышел по течению реки из пункта $A$ в пункт $B$, расположенный в $45$ км от $A$. Пробыв в пункте $B$ $3$ часа, корабль отправился назад и вернулся в пункт $A$ в $22:00$ того же дня. Определите собственную скорость корабля(в км/ч), если известно, что скорость течения реки $12$ км/ч. Ответ дайте в км/ч. }{ 18 }{323375.js}
\par{Теплоход в $1:00$ вышел по течению реки из пункта $A$ в пункт $B$, расположенный в $95$ км от $A$. Пробыв в пункте $B$ $2$ часа, теплоход отправился назад и вернулся в пункт $A$ в $15:00$ того же дня. Найдите собственную скорость теплохода(в км/ч), если известно, что скорость течения реки $14$ км/ч. Ответ дайте в км/ч. }{ 24 }{323375.js}
\par{Теплоход в $6:00$ вышел по течению реки из пункта $A$ в пункт $B$, расположенный в $4$ км от $A$. Пробыв в пункте $B$ $2$ часа, теплоход отправился назад и вернулся в пункт $A$ в $9:00$ того же дня. Определите собственную скорость теплохода(в км/ч), если известно, что скорость течения реки $3$ км/ч. Ответ дайте в км/ч. }{ 9 }{323375.js}	

\lstinputlisting[caption = 26590.js, label={lst:26590}, escapechar=|]{code/26590.js}
\subsubsection*{Примеры генерируемых задач 26590.js}   

\par{От пристани $U$ к пристани $G$, расстояние между которыми равно $224$ км, отправился с постоянной скоростью первый катер, а через $8$ часов после этого следом за ним, со скоростью на $9$ км/ч больше, отправился второй. Определите скорость первого катера, если в город $G$ оба катера прибыли одновременно. Ответ дайте в км/ч. }{ 12 }{26590.js}
\par{От пристани $X$ к пристани $U$, расстояние между которыми равно $120$ км, отправился с постоянной скоростью первый корабль, а через $5$ часов после этого следом за ним, со скоростью на $2$ км/ч больше, отправился второй. Определите скорость первого корабля, если в пункт $U$ оба корабля прибыли одновременно. Ответ дайте в км/ч. }{ 6 }{26590.js}
\par{От пристани $K$ к пристани $T$, расстояние между которыми равно $35$ км, отправился с постоянной скоростью первый теплоход, а через $7$ часов после этого следом за ним, со скоростью на $16$ км/ч больше, отправился второй. Определите скорость первого теплохода, если в село $T$ оба теплохода прибыли одновременно. Ответ дайте в км/ч. }{ 4 }{26590.js}

\lstinputlisting[caption = 5997.js, label={lst:5997}, escapechar=|]{code/5997.js}
\subsubsection*{Примеры генерируемых задач 5997.js}   

\par{Лодка в $7:00$ вышла из города $H$ в город $G$, расположенный в $54$ км от $H$. Пробыв в городе $G$ $4$ часа, лодка отправилась назад и вернулась в город $А$ в $20:00$ того же дня. Определите (в км/ч) скорость течения реки, если известно, что собственная скорость лодки равна $16$ км/ч. }{ 8 }{5997.js}
\par{Яхта в $1:00$ вышла из города $C$ в город $B$, расположенный в $45$ км от $C$. Пробыв в городе $B$ $3$ часа, яхта отправилась назад и вернулась в город $А$ в $12:00$ того же дня. Определите (в км/ч) скорость течения реки, если известно, что собственная скорость яхты равна $12$ км/ч. }{ 3 }{5997.js}
\par{Баржа в $1:00$ вышла из города $J$ в город $R$, расположенный в $117$ км от $J$. Пробыв в городе $R$ $3$ часа, баржа отправилась назад и вернулась в город $А$ в $15:00$ того же дня. Найдите (в км/ч) скорость течения реки, если известно, что собственная скорость баржи равна $22$ км/ч. }{ 4 }{5997.js}	

\lstinputlisting[caption = 506299.js, label={lst:506299}, escapechar=|]{code/506299.js}
\subsubsection*{Примеры генерируемых задач 506299.js}   

\par{Площадь трапеции вычисляется по формуле $S=\frac{a+b}{2}h$, где $a$ и $b$ - основания трапеции, $h$ - eё высота. Пользуясь этой формулой, найдите $S$, если $b=23$, $h=29$, $a=4$. }{ 391,5 }{506299.js}
\par{Площадь трапеции вычисляется по формуле $S=\frac{a+b}{2}h$, где $a$ и $b$ - основания трапеции, $h$ - eё высота. Пользуясь этой формулой, найдите $S$, если $a=4$, $b=4$, $h=29$. }{ 116 }{506299.js}
\par{Площадь трапеции вычисляется по формуле $S=\frac{a+b}{2}h$, где $a$ и $b$ - основания трапеции, $h$ - eё высота. Пользуясь этой формулой, найдите $S$, если $a=29$, $h=18$, $b=25$. }{ 486 }{506299.js}

\lstinputlisting[caption = 50703501.js, label={lst:50703501}, escapechar=|]{code/50703501.js}
\subsubsection*{Примеры генерируемых задач 50703501.js}   

\par{Если $p_1$, $p_2$ и $p_3$ - различные простые числа, то сумма всех делителей числа $p_1\\cdot p_2\\cdot p_3$ равна $(p_1+1)(p_2+1)(p_3+1)$. Найдите сумму всех делителей числа $3335=29\cdot5\cdot23$. }{ 4320 }{50703501.js}
\par{Если $p_1$, $p_2$ и $p_3$ - различные простые числа, то сумма всех делителей числа $p_1\\cdot p_2\\cdot p_3$ равна $(p_1+1)(p_2+1)(p_3+1)$. Найдите сумму всех делителей числа $1173=23\cdot17\cdot3$. }{ 1728 }{50703501.js}
\par{Если $p_1$, $p_2$ и $p_3$ - различные простые числа, то сумма всех делителей числа $p_1\\cdot p_2\\cdot p_3$ равна $(p_1+1)(p_2+1)(p_3+1)$. Найдите сумму всех делителей числа $969=3\cdot19\cdot17$. }{ 1440 }{50703501.js}

\lstinputlisting[caption = 511648.js, label={lst:511648}, escapechar=|]{code/511648.js}
\subsubsection*{Примеры генерируемых задач 511648.js}   

\par{Площадь поверхности прямоугольного параллелепипеда с ребрами $a$, $b$ и $c$ вычисляется по формуле $S=2(ab+ac+bc)$. Найдите площадь поверхности прямоугольного параллелепипеда с ребрами $9$, $8$ и $6$. }{ 348 }{511648.js}
\par{Площадь поверхности прямоугольного параллелепипеда с ребрами $a$, $b$ и $c$ вычисляется по формуле $S=2(ab+ac+bc)$. Найдите площадь поверхности прямоугольного параллелепипеда с ребрами $3$, $18$ и $28$. }{ 1284 }{511648.js}
\par{Площадь поверхности прямоугольного параллелепипеда с ребрами $a$, $b$ и $c$ вычисляется по формуле $S=2(ab+ac+bc)$. Найдите площадь поверхности прямоугольного параллелепипеда с ребрами $26$, $10$ и $28$. }{ 2536 }{511648.js}

\lstinputlisting[caption = 99605.js, label={lst:99605}, escapechar=|]{code/99605.js}
\subsubsection*{Примеры генерируемых задач 99605.js}   

\par{Первую треть трассы автомобиль ехал со скоростью $42$ км/ч, вторую треть - со скоростью $56$ км/ч, а последнюю - со скоростью $30$ км/ч. Найдите среднюю скорость автомобиля на протяжении всего пути. Ответ дайте в км/ч. }{ 40 }{99605.js}
\par{Первую треть трассы автомобиль ехал со скоростью $56$ км/ч, вторую треть - со скоростью $63$ км/ч, а последнюю - со скоростью $126$ км/ч. Найдите среднюю скорость автомобиля на протяжении всего пути. Ответ дайте в км/ч. }{ 72 }{99605.js}
\par{Первую треть трассы автомобиль ехал со скоростью $144$ км/ч, вторую треть - со скоростью $21$ км/ч, а последнюю - со скоростью $126$ км/ч. Найдите среднюю скорость автомобиля на протяжении всего пути. Ответ дайте в км/ч. }{ 48 }{99605.js}

\lstinputlisting[caption = 99604.js, label={lst:99604}, escapechar=|]{code/99604.js}
\subsubsection*{Примеры генерируемых задач 99604.js}   

\par{Путешественник переплыл море на катере со средней скоростью $228$ км/ч. Обратно он летел на спортивном самолете со скоростью $494$ км/ч. Найдите среднюю скорость путешественника на протяжении всего пути. Ответ дайте в км/ч. }{ 312 }{99604.js}
\par{Странник переплыл море на яхте со средней скоростью $279$ км/ч. Обратно он летел на спортивном самолете со скоростью $558$ км/ч. Вычислите среднюю скорость странника на протяжении всего пути. Ответ дайте в км/ч. }{ 372 }{99604.js}
\par{Странник переплыл океан на яхте со средней скоростью $92$ км/ч. Обратно он летел на спортивном самолете со скоростью $276$ км/ч. Определите среднюю скорость странника на протяжении всего пути. Ответ дайте в км/ч. }{ 138 }{99604.js}

\lstinputlisting[caption = 137382.js, label={lst:137382}, escapechar=|]{code/137382.js}
\subsubsection*{Примеры генерируемых задач 137382.js}   

\par{Найдите корень уравнения $x^2−16x=−60$ Если корней несколько, в ответе укажите их произведение. }{ 60 }{137382.js}
\par{Найдите корень уравнения $x^2−x=72$ Если корней несколько, в ответе укажите больший из них. }{ 9 }{137382.js}
\par{Найдите корень уравнения $x^2+17x=18$ Если корней несколько, в ответе укажите их произведение. }{ -18 }{137382.js}

\lstinputlisting[caption = 10001583.js, label={lst:10001583}, escapechar=|]{code/10001583.js}
\subsubsection*{Примеры генерируемых задач 10001583.js}   

\par{Найдите корень уравнения $x^2−54=15x$ Если корней несколько, в ответе укажите их произведение. }{ -54 }{10001583.js}
\par{Найдите корень уравнения $19x=x^2−120$ Если корней несколько, в ответе укажите меньший из них. }{ -5 }{10001583.js}
\par{Найдите корень уравнения $−31x=x^2+58$ Если корней несколько, в ответе укажите больший из них. }{ -2 }{10001583.js}

\lstinputlisting[caption = 311447.js, label={lst:311447}, escapechar=|]{code/311447.js}
\subsubsection*{Примеры генерируемых задач 311447.js}   

\par{Найдите корень уравнения $10x^2−17x=0$ Если корней несколько, в ответе укажите их произведение. }{ 0 }{311447.js}
\par{Найдите корень уравнения $5x^2−x=0$ Если корней несколько, в ответе укажите их сумму. }{ 0,2 }{311447.js}
\par{Найдите корень уравнения $10x^2+37x=0$ Если корней несколько, в ответе укажите меньший из них. }{ -3,7 }{311447.js}

\lstinputlisting[caption = 311438.js, label={lst:311438}, escapechar=|]{code/311438.js}
\subsubsection*{Примеры генерируемых задач 311438.js}   

\par{Найдите корень уравнения $−10x^2−9x+19=0$ Если корней несколько, в ответе укажите их произведение }{ -1,9 }{311438.js}
\par{Найдите корень уравнения $x^2+18x−19=0$ Если корней несколько, в ответе укажите меньший из них. }{ -19 }{311438.js}
\par{Найдите корень уравнения $5x2+15x+10=0$ Если корней несколько, в ответе укажите меньший из них. }{ -2 }{311438.js}

\lstinputlisting[caption = 369496.js, label={lst:369496}, escapechar=|]{code/369496.js}
\subsubsection*{Примеры генерируемых задач 369496.js}   

\par{Найдите корень уравнения $(10x−1)(16x+16)=0$ Если корней несколько, в ответе укажите их произведение. }{ -0,1 }{369496.js}
\par{Найдите корень уравнения $(2x−6)(12x−39)=0$ Если корней несколько, в ответе укажите их сумму. }{ 6,25 }{369496.js}
\par{Найдите корень уравнения $(18x+18)(14x+42)=0$ Если корней несколько, в ответе укажите меньший из них. }{ -3 }{369496.js}

\lstinputlisting[caption = 311495.js, label={lst:311495}, escapechar=|]{code/311495.js}
\subsubsection*{Примеры генерируемых задач 311495.js}   

\par{Найдите корень уравнения $x^2+315=−36x$ Если корней несколько, в ответе укажите меньший из них. }{ -21 }{311495.js}
\par{Найдите корень уравнения $−2x=x^2−440$ Если корней несколько, в ответе укажите их сумму. }{ -2 }{311495.js}
\par{Найдите корень уравнения $−34x=x^2+285$ Если корней несколько, в ответе укажите их произведение. }{ 285 }{311495.js}

\lstinputlisting[caption = 137371.js, label={lst:137371}, escapechar=|]{code/137371.js}
\subsubsection*{Примеры генерируемых задач 137371.js}   

\par{Найдите корень уравнения $14+14(−7−x)=−20+18x$ }{ -2 }{137371.js}
\par{Найдите корень уравнения $−2−30x=−21−5(14x+17)$ }{ -2,6 }{137371.js}
\par{Найдите корень уравнения $7+16x=−30−17(−3−3x)$ }{ -0,4 }{137371.js}

\lstinputlisting[caption = 338495.js, label={lst:338495}, escapechar=|]{code/338495.js}
\subsubsection*{Примеры генерируемых задач 338495.js}   

\par{Найдите корень уравнения $−3x−4−14(x−25)=13(15−x)−2$ }{ 38,25 }{338495.js}
\par{Найдите корень уравнения $−15(5−x)+13=24x−11+25(x+17)$ }{ -14 }{338495.js}
\par{Найдите корень уравнения $−17(23−x)+10=7x+27+20(x+20)$ }{ -80,8 }{338495.js}

\lstinputlisting[caption = 338480.js, label={lst:338480}, escapechar=|]{code/338480.js}
\subsubsection*{Примеры генерируемых задач 338480.js}   

\par{Найдите корень уравнения $−13x−27+(x−15)=10(9−x)+24$ }{ -78 }{338480.js}
\par{Найдите корень уравнения $−15x−29+(x−2)=−11(−17−x)+30$ }{ -9,92 }{338480.js}
\par{Найдите корень уравнения $29x−15+(x+16)=−10(15−x)−12$ }{ -8,15 }{338480.js}

\lstinputlisting[caption = 338494.js, label={lst:338494}, escapechar=|]{code/338494.js}
\subsubsection*{Примеры генерируемых задач 338494.js}   

\par{Найдите корень уравнения $(x−15)^2+(x+20)^2=2x^2$ }{ -62,5 }{338494.js}
\par{Найдите корень уравнения $(x−6)^2+(x−19)^2=2x^2$ }{ 7,94 }{338494.js}
\par{Найдите корень уравнения $2x^2=(x+9)^2+(x−18)^2$ }{ 22,5 }{338494.js}

\lstinputlisting[caption = 1111555.js, label={lst:1111555}, escapechar=|]{code/1111555.js}
\subsubsection*{Примеры генерируемых задач 1111555.js}   

\par{Найдите корень уравнения $(−10x−3)^2−4x^2=0$ Если корней несколько, в ответе укажите их сумму. }{ -0,625 }{1111555.js}
\par{Найдите корень уравнения $(3x+2)^2−4x^2=0$ Если корней несколько, в ответе укажите их сумму. }{ -2,4 }{1111555.js}
\par{Найдите корень уравнения $(−9x+8)^2−x^2=0$ Если корней несколько, в ответе укажите их сумму. }{ 1,8 }{1111555.js}

\lstinputlisting[caption = 338526.js, label={lst:338526}, escapechar=|]{code/338526.js}
\subsubsection*{Примеры генерируемых задач 338526.js}   

\par{Найдите корень уравнения $(−20−x)^2=(x−22)^2$ }{ 1 }{338526.js}
\par{Найдите корень уравнения $(−29+x)^2=(−24−x)^2$ }{ 2,5 }{338526.js}
\par{Найдите корень уравнения $(x+8)^2=(−3−x)^2$ }{ -5,5 }{338526.js}

\lstinputlisting[caption = 338482.js, label={lst:338482}, escapechar=|]{code/338482.js}
\subsubsection*{Примеры генерируемых задач 338482.js}   

\par{Найдите корень уравнения $\frac{27}{x+13}=\frac{13}{x−8}$ }{ 27,5 }{338482.js}
\par{Найдите корень уравнения $\frac{21}{x+5}=\frac{20}{x+21}$ }{ -341 }{338482.js}
\par{Найдите корень уравнения $\frac{18}{x+23}=\frac{23}{x+7}$ }{ -80,6 }{338482.js}

\lstinputlisting[caption = 1115560.js, label={lst:1115560}, escapechar=|]{code/1115560.js}
\subsubsection*{Примеры генерируемых задач 1115560.js}   

\par{Найдите корень уравнения $\frac{5x}{9}=24+\frac{5x−12}{−21}$ }{ 30,96 }{1115560.js}
\par{Найдите корень уравнения $-24+\frac{−10x+22}{−9}=\frac{30x}{−27}$ }{ 11,9 }{1115560.js}
\par{Найдите корень уравнения $\frac{17x}{−17}=20+\frac{−11x−14}{−29}$ }{ -14,85 }{1115560.js}

\lstinputlisting[caption = 311381.js, label={lst:311381}, escapechar=|]{code/311381.js}
\subsubsection*{Примеры генерируемых задач 311381.js}   

\par{Найдите корень уравнения $\frac{11}{x+7}=\frac{13}{x+18}$ }{ 53,5 }{311381.js}
\par{Найдите корень уравнения $\frac{9}{x−8}=\frac{18}{x−13}$ }{ 3 }{311381.js}
\par{Найдите корень уравнения $\frac{−6}{x+17}=\frac{-22}{x+23}$ }{ -14,75 }{311381.js}