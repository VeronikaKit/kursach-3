\documentclass[a4paper, 14pt]{extarticle}

\usepackage{ifluatex}
\usepackage{ifpdf}

\ifluatex
\usepackage{fontspec}
\defaultfontfeatures{Renderer=Basic,Ligatures={TeX}}
\setmainfont{CMU Serif}
\setsansfont{CMU Sans Serif}
\usepackage{polyglossia}
\setdefaultlanguage{russian}
\setotherlanguage{english}
\setmainfont{CMU Serif}
\newfontfamily{\cyrillicfont}{CMU Serif}
\setsansfont{CMU Sans Serif}
\newfontfamily{\cyrillicfontsf}{CMU Sans Serif}
\setmonofont{CMU Typewriter Text}
\newfontfamily{\cyrillicfonttt}{CMU Typewriter Text}
\else
\ifpdf 
\usepackage[T2A]{fontenc}
\usepackage[utf8]{inputenc}
\usepackage[english,russian]{babel}
\typeout{PDF only}
\fi
\fi

\usepackage[left=1cm,right=1cm,top=2cm,bottom=2cm]{geometry}
%%% Дополнительная работа с математикой
\usepackage{amsfonts,amssymb,amsthm,mathtools} % AMS
\usepackage{amsmath}
\usepackage{icomma} % «Умная» запятая: $0,2$ — число, $0, 2$ — перечисление

\usepackage{mathrsfs} % Красивый матшрифт

%% Перенос знаков в формулах (по Львовскому)
\newcommand*{\hm}[1]{#1\nobreak\discretionary{}
  {\hbox{$\mathsurround=0pt #1$}}{}}

%%% Работа с картинками

\usepackage{graphicx}  % Для вставки рисунков
\graphicspath{ {images/} }
\setlength\fboxsep{3pt} % Отступ рамки \fbox{} от рисунка
\setlength\fboxrule{1pt} % Толщина линий рамки \fbox{}
\usepackage{wrapfig} % Обтекание рисунков и таблиц текстом

\usepackage[dvipsnames]{xcolor}
\usepackage{verbatim}
\usepackage{hyperref}

\usepackage{listings}

\lstdefinelanguage{JavaScript}{
  keywords={let, typeof, new, true, false, catch, function, return, null, catch, switch, var, if, in, while, do, else, case, break, const},
  keywordstyle=\color{blue}\bfseries,
  ndkeywords={class, export, boolean, throw, implements, import, this,},
  ndkeywordstyle=\color{darkgray}\bfseries,
  identifierstyle=\color{black},
  sensitive=false,
  comment=[l]{//},
  morecomment=[s]{/*}{*/},
  commentstyle=\color{green}\ttfamily,
  stringstyle=\color{red}\ttfamily,
  morestring=[b]',
  morestring=[b]",
  escapechar=|
}

\lstset{
  language=JavaScript,
  extendedchars=true,
  basicstyle=\small\ttfamily,
  showstringspaces=false,
  breakatwhitespace=true,
  showspaces=false,
  numbers=left,
  numberstyle=\footnotesize,
  numbersep=9pt,
  tabsize=2,
  keepspaces=true,
  breaklines=true,
  showtabs=false,
  captionpos=b
  escapechar=|,
  frame=single ,
  commentstyle=\itshape ,
  frameshape={RYR}{Y}{Y}{RYR},
  stringstyle =\bfseries,
}

\usepackage{autobreak}

\newcommand*{\function}[2]{\texttt{\textcolor{Red}{function} \textcolor{Purple}{#1}(#2)}\linebreak}

\newcommand*{\prototype}[3][]{\texttt{\textcolor{Orange}{#2}.\textcolor{Blue}{prototype}.\textcolor{Purple}{#3} = \textcolor{Red}{function}(#1)}\linebreak}

\usepackage{titlesec}

\setcounter{secnumdepth}{4}

\titleformat{\paragraph}
{\normalfont\normalsize\bfseries}{\theparagraph}{1em}{}
\titlespacing*{\paragraph}
{0pt}{3.25ex plus 1ex minus .2ex}{1.5ex plus .2ex}

\usepackage{multicol}
\setlength{\columnsep}{1cm}

\usepackage[most]{tcolorbox}

\newtcolorbox{leftBox}{
  colback=white,colframe=black, 
  width = 0.97\linewidth,
  sharp corners = southwest,
}

\usepackage{newfloat,caption,float}
\usepackage{capt-of}

\DeclareFloatingEnvironment[
  fileext = loe,
  listname = Задача,
  name = Задача.,
  placement = H,
  within = none,
  ]{application}
\captionsetup[application]{labelfont=md}

\DeclareFloatingEnvironment[
  fileext = loe,
  listname = Рис,
  name = Рис.,
  placement = H,
  within = none,
  ]{image}
\captionsetup[image]{labelfont=md}

\usepackage{tikz,tikz-3dplot}

\renewcommand{\lstlistingname}{Листинг}% Listing -> Листинг


\usepackage{setspace}

\begin{document}


\begin{center}
	\hfill \break
	\large{МИНОБРНАУКИ РОССИИ}\\
	\footnotesize{ФЕДЕРАЛЬНОЕ ГОСУДАРСТВЕННОЕ БЮДЖЕТНОЕ ОБРАЗОВАТЕЛЬНОЕ УЧРЕЖДЕНИЕ}\\
	\footnotesize{ВЫСШЕГО ПРОФЕССИОНАЛЬНОГО ОБРАЗОВАНИЯ}\\
	\small{\textbf{«ВОРОНЕЖСКИЙ ГОСУДАРСТВЕННЫЙ УНИВЕРСИТЕТ»}}\\
	\hfill \break
	\normalsize{Математический факультет}\\
	\hfill \break
	\normalsize{Кафедра функционального анализа и операторных уравнений}\\
	\hfill\break
	\hfill \break
	\hfill \break
	\hfill \break
	\large{Применение языка JavaScript для генерации ФОС ОГЭ и ЕГЭ}\\
	\hfill \break
	\hfill \break
	\hfill \break
	\hfill \break
	\hfill \break
	\normalsize{Курсовая работа\\
		\hfill \break
		Направление 02.03.01 Математика и компьютерные науки\\

		\hfill \break
	}\\
	\hfill \break
	\hfill \break
\end{center}
\hfill \break

\normalsize{
	\begin{tabular}{cccc}
		Зав.кафедрой & \underline{\hspace{3cm}} & д.физ.-мат.н.,  проф. & М.И. Каменский    \\\\
		Обучающийся  & \underline{\hspace{3cm}} &                       & В.Д. Китаева \\\\
		Руководитель & \underline{\hspace{3cm}} & д.физ.-мат.н.,  проф. & М.И. Каменский    \\\\
	\end{tabular}
}\\
\hfill \break
\hfill \break
\begin{center} Воронеж 2025 \end{center}
\thispagestyle{empty} % выключаем отображение номера для этой страницы

% КОНЕЦ ТИТУЛЬНОГО ЛИСТА


\setstretch{1.5}

\tableofcontents



\section*{Введение}
\addcontentsline{toc}{section}{Введение}
Единый государственный экзамен (ЕГЭ)~— централизованно проводимый в Российской
Федерации экзамен в средних учебных заведениях — школах, лицеях и гимназиях,
форма проведения ГИА (Государственной Итоговой Аттестации) по образовательным программам среднего общего образования.
Служит одновременно выпускным экзаменом из школы и вступительным экзаменом в вузы.

За два года подготовки к ЕГЭ школьники сталкиваются с дефицитом заданий для подготовки. 
А учителя со списыванием ответов при решении задач экзамена учениками. 
Также в список заданий ЕГЭ были добавлены новые текстовые задания, 
количество которых для прорешивания очень мало.
Проект «Час ЕГЭ» позволяет решить все эти проблемы. «Час ЕГЭ» — компьютерный образовательный проект, 
разрабатываемый при математическом факультете ВГУ в рамках «OpenSource кластера» и предназначенный для помощи учащимся старших классов подготовиться к тестовой части единого государственного экзамена. 
Задания в «Час ЕГЭ» генерируются случайным образом по специализированным алгоритмам, называемых шаблонами, 
каждый из которых охватывает множество вариантов соответствующей ему задачи. 
Для пользователей предназначены четыре оболочки (режима работы): «Случайное задание», «Тесты на печать», «Полный тест» и «Мини-интеграция». 
«Час ЕГЭ» является полностью открытым (код находится под лицензией GNU GPL 3.0) и бесплатным. 
В настоящее время в проекте полностью реализованы тесты по математике с кратким ответом (бывшая «часть В»). 
Планируется с течением времени включить в проект тесты по другим предметам школьной программы.

Первая глава этой работы посвящена реализация алгоритмов на базе проекта Час ЕГЭ.

Вторая глава представляет решение проблемы нехватки задач ЕГЭ; шаблоны задач, написанные автором и примеры генерации.

%Цели
%всп функции для иллюстрирования
%шаблоны

%Задачи
%всп функции для иллюстрирования
%классы для 
%шаблоны



%%Программная реализация (на языке Javascript) алгоритмов
%% генерации фонда оценочных средств по математике
\section{Реализация алгоритмов на базе проекта Час ЕГЭ}
В это главе мы приводим вспомогательные функции и алгоритм написания шаблона текстовой задачи.

Функции, используемые в проекте

За 10 лет работы над проектом «Час ЕГЭ» была разработан нестандартная библиотека для упрощения многих задач. Далее представлены наиболее используемые функции из неё.
Вспомогательные функции
function sluchch(n,k,s) Возвращает случайное число от n до k с шагом s (по умолчанию 1). Эта функция используется настолько часто, что для неё была придумана сокращённая форма sl(). function slKrome(a,p1,p2,p3) Возвращает случайное число, кроме a. Если a – массив, то не содержащееся в нём; если число или строка, то не равное ему; Если функция, принимающая параметр - то не удовлетворяющее ей.
Работа с числами 
Number.prototype.chislitlx=function(p1,p2) Возвращает строку, состоящую из данного числа и подходящего падежа слова p1, при этом полученное словосочетанию стоит в падеже p2 (если не указан - именительный).
Number.prototype.pow=function(n) Возвращает число в степени n. 
Number.prototype.sqrt=function(n) Возвращает квадратный корень из числа. 
Number.prototype.sqr=function() Возвращает квадрат числа. 
Number.prototype.abs=function() Возвращает модуль числа. 
Number.prototype.floor=function() Возвращает число, округленное до целого в меньшую сторону. 
Number.prototype.ceil=function() Возвращает число, округленное до целого в большую сторону. 
Number.prototype.pm=function() Случайным образом возвращает число или ему противоположное. 
Работа со строками 
Number.prototype.toZagl=function() Возвращает исходную строку с первой заглавной буквой.
setEquationTask-отвечает за составление задач с уравнениями.
roots[] задаёт корни уравнения

Для примера возьмём задание №27482~\cite{egemath}.
\\
\textbf{Задача №27482.}
Причалы $K$ и $B$ расположены на водохранилище, расстояние между ними равно $255$ км. Моторная лодка отправилась с постоянной скоростью из $K$ в $B$. Через неделю после прибытия она отправилась тем же путём обратно со скоростью на $4$ км/ч больше прежней, сделав по пути остановку на $1$ час. В результате она затратила на обратный путь столько же времени, сколько на путь из $K$ в $B$. Определите скорость моторной лодки на пути из $K$ в $B$. Ответ дайте в км/ч.\\ 

\begin{enumerate}
    \item Выбираем задание из Открытого Банка Заданий ЕГЭ и копируем его текст.
    \item Добавляем ответ в поле answers (по умолчанию 0).
    \item Инициализируем всех необходимые переменные для задачи (вес, проценты и так далее). Присваиваем им значения при помощи функции sluchch() или slKrome() (см. главу 2). Для хранения ответа создаём отдельную переменную.
    \item Заменяем все числа в тексте на переменные (при помощи +’·’+). 
    \item Обособляем слова, которые не влияют на условия задачи. Это могут быть имена, профессии, транспорт и т.п. 
    \item Создаём переменные, которые будет отвечать за выбранные в прошлом пункте слова, и заменяем слова на переменные в тексте задачи. Выбираем их значения из массивов при помощи iz()
    \item Иногда в задании выбранные слова используются в разных падежах. Для этого в проекте существует лексический модуль. Используем на склоняемых словах функцию sklonlxkand(). Теперь необходимо указать падеж слов в задании. Также при необходимости заглавной буквы в слове используем toZagl(). Если в тексте задачи присутствуют слова, зависимые от числительных, к ним применяется функция chislitlx().
    \item Далее составляем общее решение с учетом всех переменных и особенностей задачи.
    \item Заготовка шаблона имеет следующий вид.
    \lstinputlisting[]{code/27482.js} 
\end{enumerate}

Примеры генерации задний приведены в листинге \ref{lst:109}
 %TODO: Приложение зафигачить


\section{Вклад автора в расширение каталога}\label{2sect}
В это главе решается проблема нехватки заданий для подготовки к ЕГЭ, а также приводятся аргументы в пользу программного написания шаблонов для подготовки к ЕГЭ.

\subsection{Преимущества программной генерации заданий}
На примере предыдущей задачи было явно показано превосходство шаблонов над заданиями из Открытого Банка Заданий, а именно:
\begin{enumerate}
	\item Большое количество разнообразных задач одного типа. 
	\item Простота и скорость написания шаблонов. 
    \item Невозможность нахождения учащимися ответов на задачи.
\end{enumerate}

\subsection{Вклад автора в расширение каталога}

\lstinputlisting[caption = 26586.js, label={lst:26586}, escapechar=|]{code/26586.js}
\subsubsection*{Примеры генерируемых задач 26586.js}

\par{Моторная лодка прошла против течения реки $160$ км и вернулась в пункт отправления, затратив на обратный путь на $1$ час меньше. Вычислите скорость моторной лодки в неподвижной воде, если скорость течения равна $4$ км/ч. Ответ дайте в км/ч.}{36}{26586.js}
\par{Байдарка прошла против течения реки $96$ км и вернулась в пункт отправления, затратив на обратный путь на $4$ часа меньше. Определите скорость байдарки в неподвижной воде, если скорость течения равна $1$ км/ч. Ответ дайте в км/ч.}{7}{26586.js}
\par{Яхта прошла против течения реки $180$ км и вернулась в город отправления, затратив на обратный путь на $1$ час меньше. Найдите скорость яхты в неподвижной воде, если скорость течения равна $1$ км/ч. Ответ дайте в км/ч.}{19}{26586.js}

\lstinputlisting[caption = 26588.js, label={lst:26588}, escapechar=|]{code/26588.js}
\subsubsection*{Примеры генерируемых задач 26588.js}   

\par{Байдарка проходит по течению реки до деревни назначения $18$ км и после стоянки возвращается в деревня отправления. Определите скорость течения, если скорость байдарки в неподвижной воде равна $24$ км/ч, стоянка длится $1$ час, а в деревня отправления байдарка возвращается через $3$ часа. Ответ дайте в км/ч.}{12}{26588.js}
\par{Баржа проходит по течению реки до деревни назначения $160$ км и после стоянки возвращается в деревня отправления. Определите скорость течения, если скорость баржи в неподвижной воде равна $26$ км/ч, стоянка длится $2$ часа, а в деревня отправления баржа возвращается через $15$ часов. Ответ дайте в км/ч.}{6}{326588.js}
\par{Корабль проходит по течению реки до города назначения 63 км и после стоянки возвращается в город отправления. Определите скорость течения, если скорость корабля в неподвижной воде равна 16 км/ч, стоянка длится 2 часа, а в город отправления корабль возвращается через 10 часов. Ответ дайте в км/ч.}{2}{26588.js}

\lstinputlisting[caption = 27482.js, label={lst:27482}, escapechar=|]{code/27482.js}
\subsubsection*{Примеры генерируемых задач 27482.js}   

\par{Пристани $R$ и $C$ расположены на озере, расстояние между ними равно $120$ км. Баржа отправилась с постоянной скоростью из $R$ в $C$. Через два дня после прибытия она отправилась тем же путём обратно со скоростью на $4$ км/ч больше прежней, сделав по пути остановку на $1$ час. В результате она затратила на обратный путь столько же времени, сколько на путь из $R$ в $C$. Найдите скорость баржи на пути из $R$ в $C$. Ответ дайте в км/ч.}{20}{27482.js}
\par{Пристани $J$ и $L$ расположены на водохранилище, расстояние между ними равно $168$ км. Моторная лодка отправилась с постоянной скоростью из $J$ в $L$. На следующий день после прибытия она отправилась тем же путём обратно со скоростью на $12$ км/ч больше прежней, сделав по пути остановку на $7$ часов. В результате она затратила на обратный путь столько же времени, сколько на путь из $J$ в $L$. Найдите скорость моторной лодки на пути из $J$ в $L$. Ответ дайте в км/ч.}{12}{27482.js}
\par{Пристани $M$ и $E$ расположены на водохранилище, расстояние между ними равно $136$ км. Моторная лодка отправилась с постоянной скоростью из $M$ в $E$. Через два дня после прибытия она отправилась тем же путём обратно со скоростью на $17$ км/ч больше прежней, сделав по пути остановку на $4$ часа. В результате она затратила на обратный путь столько же времени, сколько на путь из $M$ в $E$. Вычислите скорость моторной лодки на пути из $M$ в $E$. Ответ дайте в км/ч.}{17}{27482.js}

\lstinputlisting[caption = 111557.js, label={lst:111557}, escapechar=|]{code/111557.js}
\subsubsection*{Примеры генерируемых задач 111557.js}

\par{Яхта прошла против течения реки $140$ км и вернулась в пункт отправления, затратив на обратный путь на $8$ часов меньше. Найдите скорость яхты в неподвижной воде, если скорость течения равна $1$ км/ч. Ответ дайте в км/ч.}{6}{111557.js}
\par{Лодка прошла против течения реки $21$ км и вернулась в пункт отправления, затратив на обратный путь на $9$ часов меньше. Вычислите скорость лодки в неподвижной воде, если скорость течения равна $6$ км/ч. Ответ дайте в км/ч.}{8}{111557.js}
\par{Яхта прошла против течения реки $120$ км и вернулась в пункт отправления, затратив на обратный путь на $2$ часа меньше. Определите скорость яхты в неподвижной воде, если скорость течения равна $8$ км/ч. Ответ дайте в км/ч.}{32}{111557.js}

\lstinputlisting[caption = 323375.js, label={lst:323375}, escapechar=|]{code/323375.js}
\subsubsection*{Примеры генерируемых задач 323375.js}   

\par{Корабль в $10:00$ вышел по течению реки из пункта $A$ в пункт $B$, расположенный в $45$ км от $A$. Пробыв в пункте $B$ $3$ часа, корабль отправился назад и вернулся в пункт $A$ в $22:00$ того же дня. Определите собственную скорость корабля(в км/ч), если известно, что скорость течения реки $12$ км/ч. Ответ дайте в км/ч.}{18}{323375.js}
\par{Теплоход в $1:00$ вышел по течению реки из пункта $A$ в пункт $B$, расположенный в $95$ км от $A$. Пробыв в пункте $B$ $2$ часа, теплоход отправился назад и вернулся в пункт $A$ в $15:00$ того же дня. Найдите собственную скорость теплохода(в км/ч), если известно, что скорость течения реки $14$ км/ч. Ответ дайте в км/ч.}{24}{323375.js}
\par{Теплоход в $6:00$ вышел по течению реки из пункта $A$ в пункт $B$, расположенный в $4$ км от $A$. Пробыв в пункте $B$ $2$ часа, теплоход отправился назад и вернулся в пункт $A$ в $9:00$ того же дня. Определите собственную скорость теплохода(в км/ч), если известно, что скорость течения реки $3$ км/ч. Ответ дайте в км/ч.}{9}{323375.js}	

\lstinputlisting[caption = 26590.js, label={lst:26590}, escapechar=|]{code/26590.js}
\subsubsection*{Примеры генерируемых задач 26590.js}   

\par{От пристани $U$ к пристани $G$, расстояние между которыми равно $224$ км, отправился с постоянной скоростью первый катер, а через $8$ часов после этого следом за ним, со скоростью на $9$ км/ч больше, отправился второй. Определите скорость первого катера, если в город $G$ оба катера прибыли одновременно. Ответ дайте в км/ч.}{12}{26590.js}
\par{От пристани $X$ к пристани $U$, расстояние между которыми равно $120$ км, отправился с постоянной скоростью первый корабль, а через $5$ часов после этого следом за ним, со скоростью на $2$ км/ч больше, отправился второй. Определите скорость первого корабля, если в пункт $U$ оба корабля прибыли одновременно. Ответ дайте в км/ч.}{6}{26590.js}
\par{От пристани $K$ к пристани $T$, расстояние между которыми равно $35$ км, отправился с постоянной скоростью первый теплоход, а через $7$ часов после этого следом за ним, со скоростью на $16$ км/ч больше, отправился второй. Определите скорость первого теплохода, если в село $T$ оба теплохода прибыли одновременно. Ответ дайте в км/ч.}{4}{26590.js}

\lstinputlisting[caption = 5997.js, label={lst:5997}, escapechar=|]{code/5997.js}
\subsubsection*{Примеры генерируемых задач 5997.js}   

\par{Лодка в $7:00$ вышла из города $H$ в город $G$, расположенный в $54$ км от $H$. Пробыв в городе $G$ $4$ часа, лодка отправилась назад и вернулась в город $А$ в $20:00$ того же дня. Определите (в км/ч) скорость течения реки, если известно, что собственная скорость лодки равна $16$ км/ч.}{8}{5997.js}
\par{Яхта в $1:00$ вышла из города $C$ в город $B$, расположенный в $45$ км от $C$. Пробыв в городе $B$ $3$ часа, яхта отправилась назад и вернулась в город $А$ в $12:00$ того же дня. Определите (в км/ч) скорость течения реки, если известно, что собственная скорость яхты равна $12$ км/ч.}{3}{5997.js}
\par{Баржа в $1:00$ вышла из города $J$ в город $R$, расположенный в $117$ км от $J$. Пробыв в городе $R$ $3$ часа, баржа отправилась назад и вернулась в город $А$ в $15:00$ того же дня. Найдите (в км/ч) скорость течения реки, если известно, что собственная скорость баржи равна $22$ км/ч.}{4}{5997.js}	


\section*{Заключение}
\addcontentsline{toc}{section}{Заключение}
В ходе выполнеия курсовой работы за 3 курс было написано:
		      \begin{itemize}
			      \item Текстовые задачи ЕГЭ — 11 шаблонов принято.
			      \item Уравнения ОГЭ— 15 шаблонов.
		      \end{itemize}

Все добавленные в проект задания можно использовать для составления контрольных работ, проведения текущего контроля знаний учащихся, подготовки к ЕГЭ.

В будущем планируется добавить в проект большее колличество заданий ЕГЭ и ОГЭ различных типов.




\begin{thebibliography}{6}
	\bibitem{chas-ege} Тренажёр "Час ЕГЭ". – URL: https://math.vsu.ru/chas-ege/sh/katalog.html
	\bibitem{fipi}Федеральный институт педагогических измерений. – URL:  https://fipi.ru/ege/otkrytyy-bank-zadaniy-ege
	\bibitem{posobie}Момот Е. А., Арахов Н. Д. Разработка и внедрение ПО для сбора статистики результатов подготовки к ЕГЭ по математике профильного уровня //Актуальные проблемы прикладной математики, информатики и механики. – 2021. – С. 1-2.
	\bibitem{egemath}Открытый банк задач ЕГЭ по математике. Профильный уровень. – URL:  https://prof.mathege.ru/
	\bibitem{ege} Единый государственный экзамен. – URL:  https://ru.wikipedia.org/wiki/Единый\_государственный\_экзамен
	\bibitem{sdamgia}Решу ЕГЭ - Сдам ГИА. – URL: https://ege.sdamgia.ru/problem?id=27074
\end{thebibliography}


\end{document}

%Планиметрия
  %Вспомогательные функции
  %шаблоны
%Нейронные сети и шаблоны по теме «Стереометрия»
  %Современный прогресс генеративных текстовых нейросетей
  %Разработка библиотек с помощью Gpt-Chat
  %Применение ООП для разработки шаблонов
  %шаблоны

%Введение
%определение шаблона
%Основные сведения о проекте
  %1) используемые технологи
  %2)Внутренние библиотеки
  %3)Примеры шаблонов заданий(мои старые простые)
%Шаблоны на чтение графиков функции с произвольным параметром
  %вспомогательные функции добавленные к библиотеке
  %разработанные шаблоны
%Сплайны третьего порядка и шаблоны на чтение графика функции и её производной
  %понятие сплайна 3 порядка + про библиотеку
  %вспомогательные функцииЫ
  %шаблоны
%ВОЗМОЖНО ТОЛЬКО ЭТО
